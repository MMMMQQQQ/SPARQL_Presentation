\newenvironment{fullgreenverb}
{\verbbox}
{\endverbbox\par\colorbox[RGB]{204,255,204}{\parbox{\textwidth}{\theverbbox}}\par}

\newenvironment{fullblueverb}
{\verbbox}
{\endverbbox\par\colorbox[RGB]{204,229,255}{\parbox{\textwidth}{\theverbbox}}\par}
\newenvironment{fullgrayverb}
{\verbbox}
{\endverbbox\par\colorbox[RGB]{211,211,211}{\parbox{\textwidth}{\theverbbox}}\par}

\section{RDF Term Constraints}
Graph pattern matching produces a solution sequence, where each solution has a set of binding of variables to RDF terms. SPARQL \textbf{FILTER} restrict solutions o those for which the filter expression evaluates to \textbf{TRUE}
We will introduce this section using the daa graph below:
\begin{fullgreenverb}
@prefix dc:   <http://purl.org/dc/elements/1.1/> .
@prefix :     <http://example.org/book/> .
@prefix ns:   <http://example.org/ns#> .

:book1  dc:title  "SPARQL Tutorial" .
:book1  ns:price  42 .
:book2  dc:title  "The Semantic Web" .
:book2  ns:price  23 .
\end{fullgreenverb}
\subsection{Restricting the Value of Strings}
SPARQL \textbf{FILTER} functions like \textcolor{blue}{regex} can test RDF literals. \textbf{regex} matches only strig literals, \textbf{regex} ca be used to match the lexical forms of other literals by using the str function.
Query:
\begin{fullblueverb}
PREFIX  dc:  <http://purl.org/dc/elements/1.1/>
SELECT  ?title
WHERE   { ?x dc:title ?title
          FILTER regex(?title, "^SPARQL") 
        }
\end{fullblueverb}
This query will return the title which begins with "SPARQL".
Query Result:
\begin{fullgrayverb}
"SPARQL Tutorial"
\end{fullgrayverb}
Regular expression matches may be made case-insensitive with the "i" flag.
Query:
\begin{fullblueverb}
PREFIX  dc:  <http://purl.org/dc/elements/1.1/>
SELECT  ?title
WHERE   { ?x dc:title ?title
          FILTER regex(?title, "web", "i" ) 
        }
\end{fullblueverb}
This query will return the title containing the "web", "web" can be minuscule or uppercase.
Query Result:
\begin{fullgrayverb}
title
"The Semantic Web"
\end{fullgrayverb}
