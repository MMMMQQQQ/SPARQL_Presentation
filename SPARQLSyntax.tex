\newenvironment{fullgreenverb}
{\verbbox}
{\endverbbox\par\colorbox[RGB]{204,255,204}{\parbox{\textwidth}{\theverbbox}}\par}

\newenvironment{fullblueverb}
{\verbbox}
{\endverbbox\par\colorbox[RGB]{204,229,255}{\parbox{\textwidth}{\theverbbox}}\par}
\newenvironment{fullgrayverb}
{\verbbox}
{\endverbbox\par\colorbox[RGB]{211,211,211}{\parbox{\textwidth}{\theverbbox}}\par}

\section{SPARQL Syntax}
\subsection{RDF Term Syntax}
\subsubsection{Syntax for Query Variables}
A query variable is marked by the use of either "?" or "\$"; the "?" or "\$" is not part of the variable name. In a query, \$abc and ?abc identify the same variable.
\subsubsection{Synax for Blank Nodes}
\begin{fullblueverb}
[ :p "v" ] .
[] :p "v" .
\end{fullblueverb}
they have allocated a unique blank node label (here "b57") and are equivalent to:
\begin{fullblueverb}
_:b57 :p "v" .
\end{fullblueverb}
When two triple patterns have the same subject:
\begin{fullblueverb}
[ :p "v" ] :q "w" .
\end{fullblueverb}
which is equivelent to:
\begin{fullblueverb}
_:b57 :p "v" .
_:b57 :q "w" .
\end{fullblueverb}
When they have the same object:
\begin{fullblueverb}
:x :q [ :p "v" ] .
\end{fullblueverb}
which is equivelent to:
\begin{fullblueverb}
:x  :q _:b57 .
_:b57 :p "v" .
\end{fullblueverb}
\subsubsection{RDF Collections}
RDF collections can be written in triple patterns using the syntax "(element1 element2 ...)". The form "()" is an alternative for the IRI http://www.w3.org/1999/02/22-rdf-syntax-ns#nil. When used with collection elements, such as (1 ?x 3 4), triple patterns with blank nodes are allocated for the collection. \textbf{The blank node at the head of the collection can be used as a subject or object in other triple patterns}. The blank nodes allocated by the collection syntax do not occur elsewhere in the query.
\begin{fullblueverb}
(1 ?x 3 4) :p "w" .
\end{fullblueverb}
is syntactic sugar for:
\begin{fullblueverb}
 _:b0  rdf:first  1 ;
          rdf:rest   _:b1 .
_:b1  rdf:first  ?x ;
          rdf:rest   _:b2 .
_:b2  rdf:first  3 ;
      rdf:rest   _:b3 .
_:b3  rdf:first  4 ;
      rdf:rest   rdf:nil .
_:b0  :p         "w" .
\end{fullblueverb}
RDF collections can be nested and can involve other syntactic forms:
\begin{fullblueverb}
(1 [:p :q] ( 2 ) ) .
\end{fullblueverb}
is syntactc sugar for:
\begin{fullblueverb}
_:b0  rdf:first  1 ;
    rdf:rest   _:b1 .
_:b1  rdf:first  _:b2 .
_:b2  :p         :q .
_:b1  rdf:rest   _:b3 .
_:b3  rdf:first  _:b4 .
_:b4  rdf:first  2 ;
      rdf:rest   rdf:nil .
_:b3  rdf:rest   rdf:nil .
\end{fullblueverb}
\subsubsection{rdf:type}
The keywrd "a" ca be used as a predicate in a riple pattern and is an alternative for \textcolor{blue}{http://www.w3.org/1999/02/22-rdf-syntax-ns#type}. This keyword is case-sensitive.
\begin{fullblueverb}
?x  a  :Class1 .
[ a :appClass ] :p "v" .
\end{fullblueverb}
is syntactic sugar for:
\begin{fullblueverb}
?x    rdf:type  :Class1 .
_:b0  rdf:type  :appClass .
_:b0  :p        "v" .
\end{fullblueverb}